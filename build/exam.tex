\documentclass[10.5pt]{article}
\usepackage{amsmath,amssymb}
\usepackage{fontspec}
\usepackage{xcolor}
\usepackage{graphicx}
\usepackage{tabularx}
\usepackage{enumitem}
\usepackage{multicol}
\usepackage{fancyhdr}
\usepackage[many]{tcolorbox}
\usepackage[a4paper, top=16mm, bottom=20mm, left=13mm, right=13mm]{geometry}
\setlength{\parindent}{0pt}
\setlength{\parskip}{0pt}
\IfFontExistsTF{Noto Sans CJK KR}{\setmainfont{Noto Sans CJK KR}}{
  \IfFontExistsTF{Noto Sans KR}{\setmainfont{Noto Sans KR}}{
    \IfFontExistsTF{NanumGothic}{\setmainfont{NanumGothic}}{
      \IfFontExistsTF{Nanum Gothic}{\setmainfont{Nanum Gothic}}{
        \setmainfont{DejaVu Sans}
      }
    }
  }
}
\definecolor{examBlue}{HTML}{245BD1}
\definecolor{ruleGray}{gray}{0.6}
\pagestyle{fancy}
\fancyhf{}
\setlength{\headheight}{22pt}
\setlength{\headsep}{8pt}
\setlength{\footskip}{28pt}
\makeatletter
\renewcommand{\headrule}{\hbox to\headwidth{\color{examBlue}\leaders\hrule height \headrulewidth\hfill}}
\renewcommand{\footrule}{\hbox to\headwidth{\color{ruleGray}\leaders\hrule height \footrulewidth\hfill}}
\makeatother
\renewcommand{\headrulewidth}{0.8pt}
\renewcommand{\footrulewidth}{0.4pt}
\fancyhead[L]{}\fancyhead[C]{}\fancyhead[R]{\vspace*{-6pt}\textcolor{ruleGray}{\small 문항 추출기를 이용하여 제작한 시험지입니다. https://tzyping.com}}\fancyfoot[L]{}\fancyfoot[C]{\thepage}\fancyfoot[R]{https://tzyping.com}

\setlength{\columnsep}{9mm}
\setlength{\columnseprule}{0.5pt}
\setlist[enumerate,1]{label=\textcolor{examBlue}{\Large\bfseries\arabic*.}, leftmargin=*, itemsep=0.2em, topsep=0em, parsep=0pt}
\begin{document}

\thispagestyle{fancy}
\vspace*{-6mm}
\noindent\hfill{\bfseries 모의고사}
\begin{center}{\bfseries\LARGE 시험지}\end{center}
{\color{examBlue}\rule{\linewidth}{0.9pt}}
\vspace{2mm}
\renewcommand{\arraystretch}{1.35}
\begin{tabularx}{\linewidth}{@{}lX lX lX lX@{}}
이름 & \hrulefill & 날짜 & \hrulefill & 시간 & \hrulefill & 단원 & \hrulefill \\
\end{tabularx}
\vspace{8mm}
\begin{multicols}{2}
\begin{enumerate}
\item \leavevmode\begin{minipage}[t]{\linewidth}
그림과 같이 $\overline{A B}=2 \sqrt{2}, \overline{B C}=3, \overline{C A}=\sqrt{5}$ 인 삼 각형 $A B C$ 에 대하여 세 선분 $A B, B C, C A$ 위의 점을 각각 $D, E, F$ 라 하자. 삼각형 $D E F$ 의 둘레의 길이의 최솟값이 $\frac{q}{p} \sqrt{10}$ 일 때, $p+q$ 의 값은? (단, $p, q$ 는 서로소인 자연수이다.)
\par\medskip\begin{center}\includegraphics[width=0.8\linewidth]{images/img_137bc6b71292c3085cac535107750741.jpg}\end{center}\par\medskip
\vspace{0.5em}
\begin{enumerate}[label={\textcircled{\arabic*}}, itemsep=0.2em, topsep=0.2em, leftmargin=*, align=left]
\item (1) 11
\item (2) 12
\item (3) 13
\item (4) 14
\item (5) 15
\end{enumerate}
\par\vspace{6\baselineskip}
\end{minipage}
\item \leavevmode\begin{minipage}[t]{\linewidth}
전체집합 $U$ 에 대하여 두 조건 $p, q$ 의 진리집합을 각각 $P, Q$ 라 하자. 두 집합 $P, Q$ 가 그림과 같을 때, 명제 ' $p$ 이면 $\sim q$ 이다.' 가 거짓임을 보여주는 원소를 구하시오.
\par\medskip\begin{center}\includegraphics[width=0.8\linewidth]{images/img_fa5e8316713c30ae19751313c752fee8.jpg}\end{center}\par\medskip
\par\vspace{6\baselineskip}
\end{minipage}
\item \leavevmode\begin{minipage}[t]{\linewidth}
【논술형 2】최고차항의 계수가 1 인 이차함수 $f(x)$ 와 최고차항의 계수가 -1 인 이차함수 $g(x)$ 에 대하여,세 집합 $A=\{a \mid a$ 는 9 의 양의 약수 $\}$ ,$B=\{x \mid f(x)=a, a \in A\}$, $C=\{x \mid g(x)=a, a \in A\}$ 가 다음 조건을 만족시킨다.
(가)$n(A \cup B)=9, n(C)=5$
(나)$B \cap C=\{2-\sqrt{6}, 2+\sqrt{6}\}$
(다)$f(0) \in\{x \mid 0<x<3, x$ 는 실수 $\}$
다음 물음에 답하시오.[총 8.0점]
\vspace{1em}
2-1.집합 $A$ 를 원소나열법으로 나타내시오.[2.0점]
\vspace{1em}
2-2.$g(1)-f(4)$ 의 값을 구하고,그 과정을 서술하시오.[6.0점]
\par\medskip\begin{center}\includegraphics[width=0.8\linewidth]{images/img_beed6c72753b3c82cbaabe2145ef1db9.jpg}\end{center}\par\medskip
\par\vspace{6\baselineskip}
\end{minipage}
\end{enumerate}\end{multicols}
\end{document}